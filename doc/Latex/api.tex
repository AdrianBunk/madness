% This file was converted to LaTeX by Writer2LaTeX ver. 1.1.7
% see http://writer2latex.sourceforge.net for more info
\documentclass[letterpaper]{article}
\usepackage[ascii]{inputenc}
\usepackage[T1]{fontenc}
\usepackage[english]{babel}
\usepackage{amsmath}
\usepackage{amssymb,amsfonts,textcomp}
\usepackage{color}
\usepackage{array}
\usepackage{hhline}
\usepackage{hyperref}
\hypersetup{pdftex, colorlinks=true, linkcolor=blue, citecolor=blue, filecolor=blue, urlcolor=blue, pdftitle=, pdfauthor=Robert Harrison, pdfsubject=, pdfkeywords=}
% Outline numbering
\setcounter{secnumdepth}{0}
\renewcommand\thesubsection{\arabic{subsection}}
\renewcommand\thesubsubsection{\arabic{subsection}.\arabic{subsubsection}}
\renewcommand\theparagraph{\arabic{subsection}.\arabic{subsubsection}.\arabic{paragraph}}
% List styles
\newcommand\liststyleLi{%
\renewcommand\labelitemi{${\bullet}$}
\renewcommand\labelitemii{${\circ}$}
\renewcommand\labelitemiii{${\blacksquare}$}
\renewcommand\labelitemiv{${\bullet}$}
}
\newcommand\liststyleLii{%
\renewcommand\labelitemi{${\bullet}$}
\renewcommand\labelitemii{${\circ}$}
\renewcommand\labelitemiii{${\blacksquare}$}
\renewcommand\labelitemiv{${\bullet}$}
}
\newcommand\liststyleLiii{%
\renewcommand\labelitemi{${\bullet}$}
\renewcommand\labelitemii{${\circ}$}
\renewcommand\labelitemiii{${\blacksquare}$}
\renewcommand\labelitemiv{${\bullet}$}
}
\newcommand\liststyleLiv{%
\renewcommand\labelitemi{${\bullet}$}
\renewcommand\labelitemii{${\circ}$}
\renewcommand\labelitemiii{${\blacksquare}$}
\renewcommand\labelitemiv{${\bullet}$}
}
\newcommand\liststyleLv{%
\renewcommand\labelitemi{${\bullet}$}
\renewcommand\labelitemii{${\circ}$}
\renewcommand\labelitemiii{${\blacksquare}$}
\renewcommand\labelitemiv{${\bullet}$}
}
\newcommand\liststyleLvi{%
\renewcommand\labelitemi{${\bullet}$}
\renewcommand\labelitemii{${\circ}$}
\renewcommand\labelitemiii{${\blacksquare}$}
\renewcommand\labelitemiv{${\bullet}$}
}
\newcommand\liststyleLvii{%
\renewcommand\labelitemi{${\bullet}$}
\renewcommand\labelitemii{${\circ}$}
\renewcommand\labelitemiii{${\blacksquare}$}
\renewcommand\labelitemiv{${\bullet}$}
}
\newcommand\liststyleLviii{%
\renewcommand\labelitemi{${\bullet}$}
\renewcommand\labelitemii{${\circ}$}
\renewcommand\labelitemiii{${\blacksquare}$}
\renewcommand\labelitemiv{${\bullet}$}
}
\newcommand\liststyleLix{%
\renewcommand\labelitemi{${\bullet}$}
\renewcommand\labelitemii{${\circ}$}
\renewcommand\labelitemiii{${\blacksquare}$}
\renewcommand\labelitemiv{${\bullet}$}
}
% Page layout (geometry)
\setlength\voffset{-1in}
\setlength\hoffset{-1in}
\setlength\topmargin{0.7874in}
\setlength\oddsidemargin{0.7874in}
\setlength\textheight{9.062033in}
\setlength\textwidth{6.9251995in}
\setlength\footskip{26.148pt}
\setlength\headheight{0cm}
\setlength\headsep{0cm}
% Footnote rule
\setlength{\skip\footins}{0.0469in}
\renewcommand\footnoterule{\vspace*{-0.0071in}\setlength\leftskip{0pt}\setlength\rightskip{0pt plus 1fil}\noindent\textcolor{black}{\rule{0.25\columnwidth}{0.0071in}}\vspace*{0.0398in}}
% Pages styles
\makeatletter
\newcommand\ps@Standard{
  \renewcommand\@oddhead{}
  \renewcommand\@evenhead{}
  \renewcommand\@oddfoot{\hfill \thepage{}}
  \renewcommand\@evenfoot{\@oddfoot}
  \renewcommand\thepage{\arabic{page}}
}
\makeatother
\pagestyle{Standard}
\title{}
\author{Robert Harrison}
\date{2009-12-14}
\begin{document}
\section*{Summary of the MADNESS application programming interface}
THIS IS SADLY OUT OF DATE ... probably best to ignore this document until it is updated.

\subsection{Low-level sequential runtime}
\subsubsection{Shared pointers (sharedptr.h)}
Shared pointers in the spirit of Boost and the new C++ standard but with thread-safe reference counting and additional
hooks to facilitate remote access, stealing, etc. Probably needs redesigning to sit on the new standard as appropriate,
but basic functionality will remain the same.

Basic rule of thumb is when you allocate memory to immediately wrap the pointer in a \texttt{SharedPointer} (or
\texttt{SharedArray}) so that you do not have to remember to delete it and so you can freely give others the pointer
without anyone worrying about premature or double freeing of the data. Fast, thread-safe reference counting is
performed using the atomic operations.

{\ttfamily
namespace madness \{}

{\ttfamily
\ \ template {\textless}typename T{\textgreater} class SharedPtr \{}

{\ttfamily
\ \ \ \ /// Default constructor makes an null pointer}

{\ttfamily
\ \ \ \ SharedPtr(); }

{\ttfamily
\ \ \ \ /// Wrap a pointer with optional custom deleter}

{\ttfamily
\ \ \ \ explicit SharedPtr(T* ptr, void (*deleter)(T*)=0) ;}

{\ttfamily
\ \ \ \ /// Wrap a pointer with optional custom deleter \& ownership}

{\ttfamily
\ \ \ \ explicit SharedPtr(T* ptr, bool own, void (*deleter)(T*)=0);}

{\ttfamily
\ \ \ \ /// Copy constructor generates new reference to same pointer}

{\ttfamily
\ \ \ \ SharedPtr(const SharedPtr{\textless}T{\textgreater}\& s) ;}

{\ttfamily
\ \ \ \ /// Copy constructor with static type conversion}

{\ttfamily
\ \ \ \ template {\textless}typename Q{\textgreater} SharedPtr(const SharedPtr{\textless}Q{\textgreater}\& s);}

{\ttfamily
\ \ \ \ /// Destructor decrements ref. count freeing data if zero}

{\ttfamily
\ \ \ \ virtual \~{}SharedPtr();}

{\ttfamily
\ \ \ \ /// Assignment}

{\ttfamily
\ \ \ \ SharedPtr{\textless}T{\textgreater}\& operator=(const SharedPtr{\textless}T{\textgreater}\& s);}

{\ttfamily
\ \ \ \ /// Returns number of references}

{\ttfamily
\ \ \ \ int use\_count() const;}

{\ttfamily
\ \ \ \ /// Returns the value of the pointer}

{\ttfamily
\ \ \ \ T* get() const;}

{\ttfamily
\ \ \ \ /// Returns true if the SharedPtr owns the pointer}

{\ttfamily
\ \ \ \ bool owned() const; \ }

{\ttfamily
\ \ \ \ /// Cast of SharedPtr{\textless}T{\textgreater} to T* returns the value of the pointer}

{\ttfamily
\ \ \ \ operator T*() const;}

{\ttfamily
\ \ \  \ \ /// Return pointer+offset}

{\ttfamily
\ \ \ \ T* operator+(long offset) const;}

{\ttfamily
\ \ \ \ /// Return pointer-offset}

{\ttfamily
\ \ \ \ T* operator-(long offset) const;}

{\ttfamily
\ \ \ \ \ /// Dereferencing returns a reference to pointed value}

{\ttfamily
\ \ \ \ T\& operator*() const;}

{\ttfamily
\ \ \ \ \ \ /// Member access via pointer works as expected}

{\ttfamily
\ \ \ \ T* operator-{\textgreater}() const;}

{\ttfamily
\ \ \ \ /// Array indexing returns reference to indexed value}

{\ttfamily
\ \ \ \ T\& operator[](long index) const;}

{\ttfamily
\ \ \ \ /// Boolean value (test for null pointer)}

{\ttfamily
\ \ \ \ operator bool() const;}

{\ttfamily
\ \ \ \ /// Are two pointers equal?}

{\ttfamily
\ \ \ \ bool operator==(const SharedPtr{\textless}T{\textgreater}\& other) const;}

{\ttfamily
\ \ \ \ /// Are two pointers not equal?}

{\ttfamily
\ \ \ \ bool operator!=(const SharedPtr{\textless}T{\textgreater}\& other) const;}

{\ttfamily
\ \ \ \ /// Steal an un-owned reference to the pointer }

{\ttfamily
\ \ \ \ SharedPtr{\textless}T{\textgreater} steal() const;}

{\ttfamily
\ \ \};}

{\ttfamily
\ \ \ /// A SharedArray uses delete [] is used to free it}

{\ttfamily
\ \ \ template {\textless}class T{\textgreater} \ class SharedArray : public SharedPtr{\textless}T{\textgreater} \{}

{\ttfamily
\ \ \ public:}

{\ttfamily
\ \ \ \ SharedArray(T* ptr = 0);}

{\ttfamily
\ \ \ \ \ \ SharedArray(const SharedArray{\textless}T{\textgreater}\& s);}

{\ttfamily
\ \ \ \ \ /// Assignment}

{\ttfamily
\ \ \ \ SharedArray\& operator=(const SharedArray\& s);}

{\ttfamily
\ \ \};}

{\ttfamily
\}}

\subsubsection{Serialization (archive.h, bufar.h, mpiar.h, binfsar.h, vecar.h, parar.h)}
Central to MADNESS's ability to move data and computation freely around the machine is the ability to serialize all data
types with full type safety. For fundamental data types and standard containers this is already provided in
\texttt{archive.h} in which it is also described how to accomplish this for user types. See there for more info.
Various archives (serialized streams) are provided including MPI p2p messaging, binary/text sequential files, vectors,
in-memory buffers (special) and parallel archives.

Issues

\liststyleLi
\begin{itemize}
\item mpiar.h needs to be made independent of world so that it is more broadly useful and this discussion needs moving
to the low-level parallel runtime level
\item discussion of parar.h should be up at the world level
\item discussion of serialization of World and WorldObjects into bufar should be at the world level
\item need to implement type-safe serialization of ConcurrentHashMap to archive.h
\item need to discuss special role and overloading of bufar (basically user should use vecar not bufar which is reserved
for internal system use).
\item thread safety of archives (there is none!)
\end{itemize}
\subsubsection{Exceptions and assertions (madness\_exception.h)}
There is a poorly realized goal to have all exceptions generated by MADNESS be derived from \texttt{MadnessException}.
The \texttt{MADNESS\_EXCEPTION} macro wraps throwing an exception and stores the line number, function name, and source
file name in the object to aid in debugging. The \texttt{MADNESS\_ASSERTION} macro also captures the assertion code as
a string.

\subsubsection{Cpu time, wall time, processor frequency, cycle count (timers.h)}
Fast and accurate timers. Still need to understand multi-threaded and multi-core behavior.

\subsubsection{Profiling (worldprofile.h)}
MADNESS is good at breaking many profilers due to extensive use of C++ templates, inlining, 1-sided communication, etc.
For this reason we have a simple but lightweight profiling class derived from the open source SHINY project (see
sourceforge).

Issues:

\liststyleLii
\begin{itemize}
\item Document the 3 or 4 useful macros
\item Describe briefly how it works
\item Make the darn thing thread safe
\end{itemize}
\subsubsection{Pseudo-random number generation (misc/ran.h)}
A thread-safe pseudo-random number generator with single number and vector interfaces. It uses a lagged, Fibonacci
sequence with parameters described in the header file. The quality should be very high, though there are even better
generators now available and please note that although the implementation has been tested in various ways including
using Marsaglia's die-hard test the generator has not yet been employed in applications especially sensitive to the
quality of the numbers. You can have multiple independent instantiations of \texttt{Random} to obtain multiple
independent streams. The \texttt{RandomValue} and \texttt{RandomVector} functions all map to one shared default
instance.

{\ttfamily
namespace madness \{}

{\ttfamily
\ \ struct RandomState; /// Used to set/get state of the generator}

{\ttfamily
\ \ class Random \{}

{\ttfamily
\ \ public:}

{\ttfamily
\ \ \ \ \ Random(unsigned int seed = 5461);}

{\ttfamily
\ \ \ \ \ \ virtual \~{}Random();}

{\ttfamily
\ \ \ \ \ double get() ; // Returns a value uniform in [0,1)}

{\ttfamily
\ \ \ \ /// Returns a vector uniform in [0,1)}

{\ttfamily
\ \ \ \ \ \ template {\textless}typename T{\textgreater} \ void getv(int n, T * restrict v); }

{\ttfamily
\ \ \ \ /// Returns vector of random bytes in [0,256)}

{\ttfamily
\ \ \ \ \ \ void getbytes(int n, unsigned char * restrict v); }

{\ttfamily
\ \ \ \ \ \ RandomState getstate() const; \ /// Returns state of generator}

{\ttfamily
\ \ \ \ \ \ void setstate(const RandomState \&s); /// Restores state}

{\ttfamily
\ \ \ \};}

{\ttfamily
\ \ /// Random value generator with specializations for float,}

{\ttfamily
\ \ /// double, complex{\textless}float{\textgreater}, complex{\textless}double{\textgreater}, int and long}

{\ttfamily
\ \ \ template {\textless}class T{\textgreater} \ \ T RandomValue();}

{\ttfamily
\ \ /// Random vector generator with specializations for float,}

{\ttfamily
\ \ /// double complex{\textless}float{\textgreater}, complex{\textless}double{\textgreater}, int and long}

{\ttfamily
\ \ \ template {\textless}class T{\textgreater} void RandomVector(int n, T* t);}

{\ttfamily
\}}

\subsubsection{Miscellaneous (misc.h)}
{\ttfamily
namespace madness \{}

{\ttfamily
\ \ unsigned long checksum\_file(const char* filename);}

{\ttfamily
\ \ std::istream\& position\_stream(std::istream\& f, }

{\ttfamily
\ \ \ \ \ \ \ \ const std::string\& tag);}

{\ttfamily
\ \ std::string lowercase(const std::string\& s);}

{\ttfamily
\}}

\subsection{Low level parallel runtime}
This provides remote method invocation, threads, mutex, thread-safe containers, and the thread pool. At this level there
is no knowledge of the World nor perhaps even of MPI. User applications will almost certainly never need these
interfaces, unless extending low-level MADNESS functionality. The interfaces relating to threading will evolve to be
compatible with the Intel TBB (we are already adopting some of their design elements).

\subsubsection{SafeMPI -- an optionally serialized MPI wrapper (safempi.h)}
All MPI communication must go through this interface. Wraps an MPI communicator and provides all necessary MPI
functionality (please extend as necessary) plus a few small extensions. If the configure script determines that the
local MPI implementation does not safely support entry by multiple threads, all calls into all instances of
\texttt{SafeMPI} are protected by a single mutex to ensure that only one thread is ever inside the MPI library. At some
point it may be necessary to implement funneling of all calls to one thread.

{\ttfamily
namespace SafeMPI \{}

{\ttfamily
\ \ class Intracomm \{}

{\ttfamily
\ \ public:}

{\ttfamily
\ \ \ \ ... standard MPI::Intracomm methods ...}

{\ttfamily
\ \ \ \ ... extensions to simplify p2p and tree-based communication}

{\ttfamily
\ \ \};}


\bigskip

{\ttfamily
\ \ class Request \{}

{\ttfamily
\ \ \ \ public:}

{\ttfamily
\ \ ... standard MPI::Request methods ...}

{\ttfamily
\ \ \};}

{\ttfamily
\}}

Rather than repeat the entire MPI interface we refer the user to the MPI manual or \texttt{safempi.h}. 

\subsubsection{RMI -- remote method invocation (worldrmi.h)}
A single thread serves incoming active messages for all activities or Worlds on an SMP node. To be consistent with what
they actually do, this class probably needs to be renamed to active message (\texttt{AM}) and \texttt{WorldAmInterface}
renamed to \texttt{WorldRMI}. The multi-threaded processes are identified using their rank in
\texttt{MPI::COMM\_WORLD}. You are responsible for translating ranks from within your communicator to that within
\texttt{MPI::COMM\_WORLD} by getting the groups from both communicators using \texttt{Comm\_group} and then creating a
map for the ranks using \texttt{Group\_translate\_ranks}. Instantiation of the singleton is not thread safe and must be
done after initializing MPI but before it might be invoked by multiple threads. Before calling \texttt{MPI::Finalize}
you must terminate the server so that it can clean up and terminate the server thread. Failure to do so will cause
occasional SEGV upon program exit.

Messages are a contiguous buffer of which the first \texttt{RMI::HEADER\_LEN} bytes are reserved for internal use. Upon
receipt, the user payload (at \texttt{buf+HEADER\_LEN}) is guaranteed to be aligned at least on a 16-byte boundary.
When sending and receiving a message the length parameter (\texttt{nbyte}) specifies the full length of the buffer
including the header (i.e., the user data size is \texttt{nbyte-HEADER\_LEN}). The \texttt{RMI::Request} data structure
is copyable and provides at least the \texttt{Test} and \texttt{Testsome} interfaces of MPI, but note that MPI may not
be the underlying transport mechanism (on the Cray-XT we will soon be using a native Portals implementation).

Unordered messages are processed in the order of receipt that due to multiple buffers and network routes may not
correspond to the sending order. Ordered messages are processed in the order sent so that remote operations appear
sequentially consistent to a single remote thread.

Handlers should be lightweight operations and in particular should not block and to be fully safe should not send
messages (though this does work in the current MPI implementation). Anything expensive should be put as a task into the
thread pool (below). Note that there is just one server thread which is useful for simplifying maintenance of remote
data structures, but means that if you abuse it communication can back up.

{\ttfamily
namespace madness \{}

{\ttfamily
\ \ typedef void (*rmi\_handlerT)(void* buf, size\_t nbyte);}

{\ttfamily
\ \ class RMI \{}

{\ttfamily
\ \ public:}

{\ttfamily
\ \ \ \ \ \ typedef Request; // Type of structure returned by isend}

{\ttfamily
\ \ \ \ \ \ static const size\_t HEADER\_LEN;}

{\ttfamily
\ \ \ \ \ \ static const size\_t MAX\_MSG\_LEN;}

{\ttfamily
\ \ \ \ \ \ static const unsigned int ATTR\_UNORDERED;}

{\ttfamily
\ \ \ \ \ \ \ static const unsigned int ATTR\_ORDERED;}

{\ttfamily
\ \ \ \ \ \ static void begin(); // Instantiate the server}

{\ttfamily
\ \ \ \ \ \ static void end(); // Terminate the server}

{\ttfamily
\ \ \ \ \ \ static Request isend(const void* buf, size\_t nbyte, \newline
\ \ \ \ \ \ \ \ ProcessID dest, rmi\_handlerT func, \newline
\ \ \ \ \ \ \ \ unsigned int attr=ATTR\_UNORDERED); // Async. send}

{\ttfamily
\ \ \};}

{\ttfamily
\}}

\subsubsection{Atomic operations on integers (madatomic.h)}
This minimal API is likely to change to become more compatible with Intel TBB and also as we port to more machines that
might force changes. Presently all functionality is provided via macros (since the original code is old and came from
C). The argument \texttt{ptr} refers to a pointer to a \texttt{MADATOMIC\_INT}. All operations on an atomic integer
must be through the API to ensure appropriate fencing of both memory and instructions. 

{\ttfamily
MADATOMIC\_FENCE // Presently null and unused}

{\ttfamily
MADATOMIC\_INT // Type of an atomic integer (platform dependent).}

{\ttfamily
MADATOMIC\_INT\_GET(ptr) // Read value of atomic integer}

{\ttfamily
MADATOMIC\_INT\_SET(ptr,value) // Write value to atomic integer}

{\ttfamily
MADATOMIC\_INT\_INC(ptr) // Atomic increment of atomic integer}

{\ttfamily
MADATOMIC\_INT\_DEC\_AND\_TEST(ptr) // Decrement and test atomic integer}

{\ttfamily
\ \ \ \ \ \ \ \ \ \ \ \  \ \ // Returns true if result is zero}

{\ttfamily
MADATOMIC\_INITIALIZE(val) // For static initialization}


\bigskip

Static initialization of an integer (i.e., where it is declared) must be performed as follows

{\ttfamily
MADATOMIC\_INT atom = MADATOMIC\_INITIALIZE(value)}


\bigskip

\subsubsection{Mutex, MutexReaderWriter, ScopedMutex, MutexWaiter (thread.h)}
Simple locking mechanisms based upon either atomic memory operations or Pthread mutexes. The locks are spin-locks with a
rudimentary back-off algorithm implemented by \texttt{MutexWaiter}. Over subscribing processors might cause performance
problems without care. \texttt{Mutex} provides a simple lock. \texttt{MutexReaderWriter} provides a non-exclusive read
lock and an exclusive write lock. \texttt{ScopedMutex} wraps an existing \texttt{Mutex} to protect an entire scope
following the rules of the lifetime of C++ objects. Mutexes of any kind cannot be copied or assigned. The locks are not
recursive -- i.e., if a thread has a lock and attempts to lock it again the results are undefined (probably deadlock).

{\ttfamily
namespace madness \{}

{\ttfamily
\ \ class Mutex \{}

{\ttfamily
\ \ public:\newline
\ \ \ \ Mutex();}

{\ttfamily
\ \ \ \ void lock() const;}

{\ttfamily
\ \ \ \ void unlock() const;}

{\ttfamily
\ \ \ \ bool try\_lock() const; // Returns true if acquired}

{\ttfamily
\ \ \ \ virtual \~{}Mutex();}

{\ttfamily
\ \ \};}


\bigskip

{\ttfamily
\ \ class MutexReaderWriter \{}

{\ttfamily
\ \ public:}

{\ttfamily
\ \ \ \ \ static const int NOLOCK;}

{\ttfamily
\ \ \ \ static const int READLOCK;}

{\ttfamily
\ \ \ \ static const int WRITELOCK;}

{\ttfamily
\ \ \ \ bool try\_read\_lock() const;}

{\ttfamily
\ \ \ \ bool try\_write\_lock() const;}

{\ttfamily
\ \ \ \ bool try\_lock(int lockmode) const;}

{\ttfamily
\ \ \ \ bool try\_convert\_read\_lock\_to\_write\_lock() const;}

{\ttfamily
\ \ \ \ void read\_lock() const;}

{\ttfamily
\ \ \ \ void write\_lock() const;}

{\ttfamily
\ \ \ \ void lock(int lockmode) const;}

{\ttfamily
\ \ \ \ void read\_unlock() const;}

{\ttfamily
\ \ \ \ void write\_unlock() const;}

{\ttfamily
\ \ \ \ void unlock(int lockmode) const;}

{\ttfamily
\ \ \ \ void convert\_read\_lock\_to\_write\_lock() const;}

{\ttfamily
\ \ \ \ void convert\_write\_lock\_to\_read\_lock() const;}

{\ttfamily
\ \ \ \ virtual \~{}MutexReaderWriter();}

{\ttfamily
\ \ \};}


\bigskip

{\ttfamily
\ \ class ScopedMutex \{}

{\ttfamily
\ \ public:}

{\ttfamily
\ \ \ \ ScopedMutex(Mutex\&); // Constructor acquires lock}

{\ttfamily
\ \ \ \ ScopedMutex(Mutex*); // Constructor acquires lock\newline
\ \ \ \ virtual \~{}ScopedMutex(); // Destructor releases lock}

{\ttfamily
\ \ \};}


\bigskip

{\ttfamily
\ \ /// Attempt to acquire two locks without blocking while \newline
\ \ /// holding either one}

{\ttfamily
\ \ bool try\_two\_locks(const Mutex\& m1, const Mutex\& m2);}

{\ttfamily
\}}


\bigskip

\subsubsection{Threads (thread.h)}
This provides simple wrappers around POSIX standard Pthread calls to simplify creation of detached system scheduled
threads. It is probable that this interface will change either to match Intel TBB or the new C++ standard. The first
wrapper provides a base class that enables object instances to be associated with threads. The second is probably not
worth discussing. The derived class must call \texttt{ThreadBase::start()} to commence the thread which executes the
virtual run method. To terminate the thread the run method can return or the thread can invoke
\texttt{ThreadBase::exit()}. Another thread can cancel the thread by using \texttt{ThreadBase::cancel()};

{\ttfamily
namespace madness \{}

{\ttfamily
\ \ class ThreadBase \{}

{\ttfamily
\ \ public:}

{\ttfamily
\ \ \ \ ThreadBase(); // Must invoke start() to start the thread.}

{\ttfamily
\ \ \ \ virtual void run() = 0; // Implement this to do work}

{\ttfamily
\ \ \ \ void start(); // Start the thread running}

{\ttfamily
\ \ \ \ static void exit(); // Call for premature exit from run()}

{\ttfamily
\ \ \ \ const pthread\_t\& get\_id(); // Get pthread id (if running)}

{\ttfamily
\ \ \ \ int get\_pool\_thread\_index() const; // Index in ThreadPool or -1}

{\ttfamily
\ \ \ \ int cancel() const; // Cancel this thread}

{\ttfamily
\ \ \ \ virtual \~{}ThreadBase();}

{\ttfamily
\ \ \ \ \};}

{\ttfamily
\}}


\bigskip

Mmm ... I see that the ThreadBase destructor does not cancel the thread if it is still running ... probably should? Seem
to recall an issue here but it seems bad to delete the object and leave the thread running. 

\subsubsection{Tasks, task attributes and the thread pool (thread.h)}
The primary (only?) source of SMP concurrency inside MADNESS comes from inserting tasks into the task queue for
execution by the pool of threads. The class is a singleton. The one pool of threads serves all activities and worlds
within the SMP node so as to avoid oversubscribing the processors and to facilitate (eventually) switching between
cache-aware and cache-oblivious computation. Presently we just have cache oblivious. A task is derived from
\texttt{PoolTaskInterface} and is submitted to the queue by invoking \texttt{ThreadPool::add()} which takes ownership
of the task and deletes it upon completion. Tasks have attributes that presently can specify high priority (task is
inserted at the front of the queue instead of the rear), generation of additional parallelism, and stealable. The last
two are presently ignored. As for the RMI class, instantiate the singleton with \texttt{ThreadPool::begin()} before
adding tasks to avoid a race condition. The number of threads may be specified as an argument or via the environment
variable \texttt{POOL\_NTHREAD} (probably to be renamed).

{\ttfamily
namespace madness \{}

{\ttfamily
\ \ class TaskAttributes \{}

{\ttfamily
\ \ public:}

{\ttfamily
\ \ \ \ static const unsigned long GENERATOR;}

{\ttfamily
\ \ \ \ static const unsigned long STEALABLE;}

{\ttfamily
\ \ \ \ static const unsigned long HIGHPRIORITY;}

{\ttfamily
\ \ \ \ TaskAttributes(unsigned long flags = 0);}

{\ttfamily
\ \ \ \ bool is\_generator() const;}

{\ttfamily
\ \ \ \ bool is\_stealable() const;}

{\ttfamily
\ \ \ \ bool is\_high\_priority() const;}

{\ttfamily
\ \ \ \ void set\_generator (bool generator\_hint);}

{\ttfamily
\ \ \ \ void set\_stealable (bool stealable);}

{\ttfamily
\ \ \ \ void set\_highpriority (bool hipri);}

{\ttfamily
\ \ \ \ static TaskAttributes generator();}

{\ttfamily
\ \ \ \ static TaskAttributes hipri();}

{\ttfamily
\ \ \};}


\bigskip

{\ttfamily
\ \ class PoolTaskInterface : public TaskAttributes \{}

{\ttfamily
\ \ public:}

{\ttfamily
\ \ \ \ PoolTaskInterface();}

{\ttfamily
\ \ \ \ explicit PoolTaskInterface(const TaskAttributes\& attr);}

{\ttfamily
\ \ \ \ virtual void run() = 0; // Implement to do work}

{\ttfamily
\ \ \ \ virtual \~{}PoolTaskInterface() \{\}}

{\ttfamily
\ \ \ \};}


\bigskip

{\ttfamily
\ \ class ThreadPool \{}

{\ttfamily
\ \ public:}

{\ttfamily
\ \ \ \ static void begin(nthread=0); // Instantiate the pool}

{\ttfamily
\ \ \ \ static void add(PoolTaskInterface*); // Add task to queue}

{\ttfamily
\ \ \ \ static void add(const std::vector{\textless}PoolTaskInterface*{\textgreater}\&);}

{\ttfamily
\ \ \ \ \~{}ThreadPool() \{\};}

{\ttfamily
\ \ \ \ \};}

{\ttfamily
\}}

\subsubsection{Callbacks and tracking dependencies (worlddep.h)}
MADNESS's internal execution is at least in part software data flow with dependencies between tasks being managed by
counting dependencies. Callbacks are used to count dependencies and to perform the action(s) taken when all
dependencies are satisfied. The present rudimentary interface is thread safe and may be extended to provide greater
functionality.

{\ttfamily
namespace madness \{}

{\ttfamily
\ \ class CallbackInterface \{}

{\ttfamily
\ \ public:}

{\ttfamily
\ \ \ \ virtual void notify() = 0;}

{\ttfamily
\ \ \ \ virtual \~{}CallbackInterface()\{\};}

{\ttfamily
\ \ \};}


\bigskip

{\ttfamily
\ \ class DependencyInterface : public CallbackInterface \{}

{\ttfamily
\ \ public:}

{\ttfamily
\ \ \ \ DependencyInterface(int ndep = 0);}

{\ttfamily
\ \ \ \ int ndep() const; // Returns no. of unsatisfied dependencies}

{\ttfamily
\ \ \ \ bool probe() const; // Returns true if ndep() == 0}

{\ttfamily
\ \ \ \ void notify(); // Callback for dependency being satisfied}

{\ttfamily
\ \ \ \ void inc(); // Increment the number of dependencies}


\bigskip

{\ttfamily
\ \ \ \ // Registers a callback for when ndepend=0.}

{\ttfamily
\ \ \ \ // If ndepend == 0, the callback is immediately invoked.}

{\ttfamily
\ \ \ \ // ADDING A CALL BACK IS NOT PRESENTLY THREAD SAFE.}

{\ttfamily
\ \ \ \ void register\_callback(CallbackInterface* callback);}


\bigskip

{\ttfamily
\ \ \ \ \ \ virtual \~{}DependencyInterface();}

{\ttfamily
\ \ \ \ \};}

\}

\subsubsection{Ranges -- specifying parallel iteration}
Ranges automate the expression and decomposition of an index space for parallel iteration. Our present capability is
extremely limited and crude when compared with the TBB, but hile our interface adopts some of the TBB conventions, we
use futures both internally and externally to facilitate the generation of more parallelism in the user code.

{\ttfamily
namespace madness \{}

{\ttfamily
\ \ template {\textless}typename iteratorT{\textgreater} class Range \{}

{\ttfamily
\ \ public:}

{\ttfamily
\ \ \ \ typedef iteratorT iterator;}

{\ttfamily
\ \ \ \ Range(const iterator\& start, const iterator\& finish, \newline
\ \ \ \ \ \ \ \ int chunksize=-1);}


\bigskip

{\ttfamily
\ \ \ \ \ \ Range(const Range\& r);}

{\ttfamily
\ \ \ \ Range(Range\& r, const Split\& split);}

{\ttfamily
\ \ \ \ size\_t size() const;}

{\ttfamily
\ \ \ \ \ \ bool empty() const;}

{\ttfamily
\ \ \ \ const iterator\& begin() const;}

{\ttfamily
\ \ \ \ const iterator\& end() const;}

{\ttfamily
\ \ \};}

{\ttfamily
\}}

\subsubsection[Futures {}-- managing latency and dependencies (future.h and worldref.h)]{Futures -- managing latency
and dependencies (future.h and worldref.h)}
The presently world level

\liststyleLiii
\begin{itemize}
\item {\ttfamily
Future}
\item {\ttfamily
RemoteReference}
\end{itemize}
classes should be stripped of references to world and pushed down here so that we have a complete parallel programming
environment independent of the World stuff.

\subsubsection{Concurrent hash map (worldhashmap.h)}
A thread-safe hash map that is a drop in replacement for the GNU \texttt{hash\_map} and new C++
\ \texttt{unordered\_map}, but has the additional features

\liststyleLiv
\begin{itemize}
\item Concurrent deletions, insertions or modifications of entries do not invalidate iterators or pointers to entries
(i.e., will not make them point to junk unless someone actually deletes the entry you are pointing at).
\item Intel TBB-style accessor interfaces to provide full thread-safe access to entries with either read or write
mutexes.
\end{itemize}
Issues:

\liststyleLv
\begin{itemize}
\item Should be renamed concurhashmap.h
\item Include class prototype here
\end{itemize}

\bigskip

\subsection{World level runtime}
The \texttt{World} class encapsulates the entire global runtime environment for either all or a sub-group of processes.
It wraps a single MPI communicator. The main capabilities for the global runtime are

\liststyleLvi
\begin{itemize}
\item One-sided active messaging including remote task creation
\item Local and global fencing of AM and tasks (termination detection)
\item Global names (id) for objects
\item Globally addressable data structures including hash tables and arrays (still coming)
\item Parallel serialization of data 
\end{itemize}

\bigskip

\subsubsection{World (world.h)}
????

\subsubsection{World MPI interface (worldmpi.h)}
This is really a convenience for you to avoid having to first get the safe MPI communicator and then use it and is also
a hook for future instrumentation.

\subsubsection[World active message interface (worldam.h)]{World active message interface (worldam.h)}
This service sits on top of the global RMI described above (note that correct translation of ranks between communicators
is not yet being performed ... i.e., only works for COMM\_WORLD). It adds to the RMI interface automatic send buffer
management, eliminates user concern with the RMI header, throttles the number of outstanding send operations, and
propagates state (the world, source process, message size) by encapsulatation in the class \texttt{AmArg} that also
facilitates packing and unpacking of the message buffer using serialization.

A World active message handler has this type

{\ttfamily
namespace madness \{}

{\ttfamily
\ \ typedef void (*am\_handlerT)(const AmArg\&);}

{\ttfamily
\}}

\paragraph{AmArg (worldam.h)}
This class and its helper functions greatly simplify packing and unpacking data from message buffers.


\bigskip

{\ttfamily
namespace madness \{}

{\ttfamily
\ \ class AmArg \{}

{\ttfamily
\ \ public:}

{\ttfamily
\ \ \ \ AmArg();}

{\ttfamily
\ \ \ \ unsigned char* buf() const; // Returns pointer to payload}

{\ttfamily
\ \ \ \ size\_t size() const; // Returns size of payload}

{\ttfamily
\ \ \ \ \ \ template {\textless}typename T{\textgreater} \ }

{\ttfamily
\ \ \ \  \ BufferInputArchive operator\&(T\& t) const; // Deserialize}

{\ttfamily
\ \ \ \ template {\textless}typename T{\textgreater}}

{\ttfamily
\ \ \ \  \ BufferOutputArchive operator\&(const T\& t) const; // Serialize}

{\ttfamily
\ \ \ \ ProcessID get\_src() const; // Source of incoming msg.}

{\ttfamily
\ \ \ \ World* get\_world() const; // World of incoming msg.}

{\ttfamily
\ \ \};}


\bigskip

{\ttfamily
\ \ // Allocates a new AmArg with nbytes of user data}

{\ttfamily
\ \ \ inline AmArg* alloc\_am\_arg(std::size\_t nbyte);}


\bigskip

{\ttfamily
\ \ // Duplicates an AmArg}

{\ttfamily
\ \ \ inline AmArg* copy\_am\_arg(const AmArg\& arg);}


\bigskip

{\ttfamily
\ \ // Frees an AmArg allocated with alloc\_am\_arg or copy\_am\_arg}

{\ttfamily
\ \ inline void free\_am\_arg(AmArg* arg);}


\bigskip

{\ttfamily
\ \ // Serialize one argument into a new AmArg}

{\ttfamily
\ \ template {\textless}typename A{\textgreater}}

{\ttfamily
\ \ inline AmArg* new\_am\_arg(const A\& a);}


\bigskip

{\ttfamily
\ \ \ // Serialize two arguments in a new AmArg}

{\ttfamily
\ \ template {\textless}typename A, typename B{\textgreater}}

{\ttfamily
\ \ inline AmArg* new\_am\_arg(const A\& a, const B\& b);}


\bigskip

{\ttfamily
\ \ // Ditto for up to 10 arguments}

{\ttfamily
\}}


\bigskip

\paragraph{WorldAmInterface (worldam.h)}
This class is embedded inside World and you probably will not instantiate it separately.

{\ttfamily
namespace madness \{}

{\ttfamily
\ \ class WorldAmInterface \{}

{\ttfamily
\ \ \ \ WorldAmInterface(World\& world);}


\bigskip

{\ttfamily
\ \ \ \ \ \ // Ensures all local AM operations have completed}

{\ttfamily
\ \ \ \ \ \ inline void fence();}


\bigskip

{\ttfamily
\ \ \ \ // Sends non-blocking message -- user responsible for managing\newline
\ \ \ \ // completion and freeing buffer}

{\ttfamily
\ \ \ \ \ \ RMI::Request isend\_(ProcessID dest, am\_handlerT op, \newline
\ \ \ \ \ \ \ \ const AmArg* arg, int attr=RMI::ATTR\_ORDERED);}


\bigskip

{\ttfamily
\ \ \ \ \ \ // Sends manged non-blocking non-blocking active message.\newline
\ \ \ \ // The buffer will be freed when the send is complete}

{\ttfamily
\ \ \ \ \ \ void send(ProcessID dest, am\_handlerT op, }

{\ttfamily
\ \ \ \ \ \ \ \ const AmArg* arg, int attr=RMI::ATTR\_ORDERED);}

{\ttfamily
\ \ \ \ \};}

{\ttfamily
\}}


\bigskip

The \texttt{fence()}method ensures local completion. After the call, all managed outgoing messages will have been sent
(though not necessarily processed by the remote end) and all received incoming active messages will have been executed.

\subsubsection{World global operations (worldgop.h)}
Except for fence these were originally needed to avoid deadlock in the single threaded code. Are still useful if MPI is
single-threaded (most common) or if we expose asynchronous global operations.

\liststyleLvii
\begin{itemize}
\item Add discussion of what global fence means and its cost
\end{itemize}

\bigskip

\subsubsection{World task interface (world\_task\_queue.h)}
Provides submission of local and remote tasks using serialization to move arguments, and futures with callbacks to
manage dependencies. Counts number of pending tasks to enable fencing.

\subsubsection{World objects (world\_object.h)}
Provides global names for single replicated objects so that can send messages to instance of the logically same object
on a different processor without having to keep track of pointers etc.

\subsubsection[World container (worlddc.h)]{World container (worlddc.h)}
A globally addressable hash table with one-sided access. Can send messages to either the container (using inherited
WorldObj methods) or to items in the container. The latter is a centrally important ability because

\liststyleLviii
\begin{itemize}
\item it enables non-process-centric computing -- it eliminates reference to process number and hence \ virtualizes how
and where computation occurs
\item it automatically gives you a write lock on the item via a non-const accessor
\end{itemize}
Issues

\liststyleLix
\begin{itemize}
\item Gotta be careful when acquiring other locks
\item Need to modify send protocol to break call chains


\bigskip
\end{itemize}
\end{document}
