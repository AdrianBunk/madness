\documentclass[letterpaper]{book}
%\documentclass[letterpaper]{article}
\usepackage{hyperref}
\usepackage{amsmath}

%%\hypersetup{pdftex, colorlinks=true, linkcolor=blue, citecolor=blue, filecolor=blue, urlcolor=blue, pdftitle=, pdfauthor=Robert Harrison, pdfsubject=, pdfkeywords=}
%%% Outline numbering
%%\setcounter{secnumdepth}{3}
%%\renewcommand\thesection{\arabic{section}}
%%\renewcommand\thesubsection{\arabic{section}.\arabic{subsection}}
%%\renewcommand\thesubsubsection{\arabic{section}.\arabic{subsection}.\arabic{subsubsection}}

%%% Page layout (geometry)
%%\setlength\voffset{-1in}
%%\setlength\hoffset{-1in}
%%\setlength\topmargin{0.7874in}
%%\setlength\oddsidemargin{0.7874in}
%%\setlength\textheight{8.825199in}
%%\setlength\textwidth{6.9251995in}
%%\setlength\footskip{0.6in}
%%\setlength\headheight{0cm}
%%\setlength\headsep{0cm}
%%% Footnote rule
%%\setlength{\skip\footins}{0.0469in}
%%\renewcommand\footnoterule{\vspace*{-0.0071in}\setlength\leftskip{0pt}\setlength\rightskip{0pt plus 1fil}\noindent\textcolor{black}{\rule{0.25\columnwidth}{0.0071in}}\vspace*{0.0398in}}

% Paragraph formatting
\setlength{\parindent}{0pt}
\setlength{\parskip}{2ex plus 0.5ex minus 0.2ex}

\begin{document}

% Title Page
\title{Getting Started with Numerical Computation in MADNESS}
\date{Last Modification: 07/10/2010}
\maketitle

% Copyright Page
\pagestyle{empty}
\null\vfill
\noindent
This file is part of MADNESS.


Copyright (C) 2007, 2010 Oak Ridge National Laboratory

This program is free software; you can redistribute it and/or modify it under the terms of the GNU General Public
License as published by the Free Software Foundation; either version 2 of the License, or(at your option) any later
version.

This program is distributed in the hope that it will be useful, but WITHOUT ANY WARRANTY; without even the implied
warranty of MERCHANTABILITY or FITNESS FOR A PARTICULAR PURPOSE. See the GNU General Public License for more details.

You should have received a copy of the GNU General Public License along with this program; if not, write to the Free
Software Foundation, Inc., 59 Temple Place, Suite 330, Boston, MA 02111-1307 USA

For more information please contact:
\begin{quote}							
Robert J. Harrison 				\\
Oak Ridge National Laboratory 	\\
One Bethel Valley Road 			\\
P.O. Box 2008, MS-6367			\\
Oak Ridge, TN 37831				\\
								\\
email: harrisonrj@ornl.gov 		\\
tel: 865-241-3937				\\
fax: 865-572-0680	
\end{quote}		
\newpage


% Table of Contents Pages
\clearpage
\setcounter{page}{1}
\pagenumbering{roman}

\setcounter{tocdepth}{10}
\renewcommand\contentsname{Table of Contents}
\tableofcontents


\clearpage
\setcounter{page}{1}
\pagenumbering{arabic}

\chapter{Getting Started}

This document aims to satisfy the needs of first-time users of the MADNESS numerical programming environment.
Programmers interested in using the parallel-programming environment should refer to the parallel runtime document.

This is the first official release of MADNESS and we know there are lots of things missing and that some things do not
work as well as we might hope. However, things will improve faster if you get involved. \ Minimally, let us know of
deficiencies, problems, and feature requests. The next step is to contribute documentation, examples, bug fixes or even
new functionality. We aren't just a community project, we are a community, and you are welcome.

\section{What is MADNESS and what can you do with it?}
MADNESS stands for multiresolution adaptive numerical environment for scientific simulation. It is trying to address the
following issues

\begin{itemize}
\item raising the level of composition of scientific applications, making it faster/easier to craft robust/correct
solutions to new problems,
\item computing using functions and operators instead of just numbers, vectors and matrices,
\item providing fast and accurate solutions for a variety of differential and integral equations in 1 to 6 dimensions
(and perhaps higher), and
\item facilitating use of massively parallel computer resources by a wider audience.
\end{itemize}

In MADNESS, your code is written in terms of functions and operators, using the C++ language, and for this reason it can
be thought of as a basis-free method. There is, of course, an underlying representation and approximation using bases
and grids, and they adapt and refine automatically to satisfy the requested precision -- but you do not have to think
about this until efficiency or memory use become concerns, if ever.

The numerical operations can be regarded as a finite precision equivalent of the mathematical calculus used to express
your equations. E.g., the code corresponding to 

\begin{equation}
\int \left(\frac{1}{2}\left(\frac{d\psi }{dx}\right)^{2}+V(x)\psi (x)^{2}\right)\mathit{dx}
\end{equation}
in three dimensions is just
\begin{quote}
\begin{verbatim}
	real_function_3d dpsidx = Dx(psi);
	double result = 0.5*inner(dpsidx,dpsidx) + inner(V(x),psi*psi);	
\end{verbatim}
\end{quote}
where \texttt{psi} is a 3-D function and \texttt{Dx(psi)} takes the derivative of \texttt{psi} in the  $x$ direction
(see below for how to make the derivative operator). The \texttt{inner} method computes the  $L_{2}$ inner product of
the functions.

Within MADNESS you can apply both differential operators and integral operators. Important and common convolution
operators with physically relevant Green functions (e.g., Coulomb, bound-state Helmholtz, free-particle quantum
propagator) are built in. Many physical problems can be restated in integral form with huge benefits in ease and
accuracy of solution. Finally, MADNESS hopefully excels at obtaining high-accuracy solutions, efficient computation in
many dimensions, and use of massively parallel computers.

MADNESS is not good at everything. In particular, if you have complicated boundary conditions, highly oscillatory
functions, or only need low precision, then other tools might be more appropriate.

In the following, we introduce MADNESS using problems of increasing complexity to accomplish standard tasks and to
discuss topics that are central to effective use of MADNESS.

\section{Basics}
Documentation and instructions exists in several places

\begin{itemize}
\item the \texttt{doc} directory,
\item the \href{http://www.doxygen.org/}{\texttt{doxygen}} generated documentation, and
\item the \href{http://code.google.com/p/m-a-d-n-e-s-s/}{project wiki}.
\end{itemize}
For basic computing with MADNESS functions, the only header file that you should need to include is
\texttt{{\textless}mra/mra.h{\textgreater}}. Advanced programs may need a few others. You should include the directory
in the include path to avoid possible conflicts with similarly named files from other packages. \ In the following
documentation \texttt{trunk} will denote the top-level directory of the MADNESS distribution.

\section{From math to C++ to numerical computation}
This section shows you how to use MADNESS and evaluate a simple mathematical expression, notably this one dimensional
integral.

\begin{equation}
\int _{0}^{10}\sin (x)\mathit{dx}=1-\cos (10)=1.8390715\ldots 
\end{equation}
The corresponding code is in \texttt{trunk/src/apps/examples/sininteg.cc}, and we will go through this first example in
gory detail. The main steps will be

\begin{enumerate}
\item translating the function from math to C++ to be called by MADNESS
\item initializing the MADNESS parallel runtime
\item initializing the MADNESS numerical environment
\item generating a numerical representation from the C++ function
\item computing the integral
\item printing the result
\item terminating the MADNESS parallel runtime
\item compiling your program
\item running your program
\end{enumerate}
\subsection{Step 1 -- translating from Math to C++}
MADNESS generates a numerical representation of a function ( $f(x)$) by projecting into its internal, orthonormal basis 
$\{\phi _{li}^{n}(x)\}$, see the implementation notes for more detail. This involves computing many integrals of the
form 

\begin{equation}
s_{li}^{n}=\int \phi _{li}^{n}(x)f(x)\mathit{dx}
\end{equation}
using adaptive numerical quadrature. Thus, MADNESS has to be able to evaluate your function at an arbitrary point, and
so you must provide it with a C++ implementation of your function using a standard interface. MADNESS passes to your
function the coordinates (an array of the appropriate dimension) and you return the value. For simple functions, a C++
function suffices whereas more complicated stuff might require a C++ class (see a subsequent example).

For this example, we wish to evaluate the function  $\sin (x)$ where  $x$ is a 1-D coordinate. \ The example code looks
like
\begin{quote}
	\begin{verbatim} 
	double myf(const coord_1d& r) {
		return std::sin(r[0]);
	}
	\end{verbatim}
\end{quote}	

If your function were in 3-D it would be passed a \texttt{coord\_3d} that would have 3 elements (0, 1, and 2) that you
might interpret as  $(x,y,z)$ or  $(r,\theta ,\phi )$ or something else as defined by your problem. A very useful
capability of both Maple and Mathematica is to generate C from expressions or functions, though the resulting code
often requires a little cleanup. 

\textit{Important:}\textit{ }If your function contains discontinuities (in value or derivatives), noise, or
singularities, special care is needed. This is discussed in more detail below, but the simple rule of thumb is your
function should be accurate nearly to machine precision even if you only want to compute to much lower precision with
MADNESS.

\textit{Important:} MADNESS will call your function from multiple threads in order to use multi-core processors
efficiently, so the code for your function should be thread safe. It should not modify static (Fortran keyword
\texttt{save}) or global (Fortran common blocks or modules) data; reading from such locations is fine. Note that this
constraint applies to all code invoked by your function, including math, BLAS, and linear algebra libraries and for
some machines/compilers it requires special options be used (the MADNESS makefiles look after this). If you suspect a
thread problem, try running with the environment variable \texttt{MAD\_NUM\_THREADS} set to one.

\textit{Important: }If you are using the AMD math library ACML, do not set the environmental variable
\texttt{OMP\_NUM\_THREADS} (or set it to one).

\subsection{Steps 2, 3 and 7 -- initializing and finalizing the MADNESS environment}
MADNESS has its own parallel runtime environment that is fully compatible with the Message Passing Interface (MPI) .
Just as for MPI, the MADNESS runtime must be initialized at the start and cleaned up at the end. Here is the simplest
parallel program using MADNESS.

\begin{quote}
	\begin{verbatim} 
	#define WORLD_INSTANTIATE_STATIC_TEMPLATES
	#include <madness/mra/mra.h>
	using namespace madness;
	
	int main(int argc, char** argv) {
	    initialize(argc, argv);
	    World world(MPI::COMM_WORLD);
	    startup(world, argc, argv);
		
	    // Do something useful here!
	    
	    finalize();
	    return 0;
	}
	\end{verbatim}
\end{quote}	

The \texttt{World} object in MADNESS is similar to an MPI communicator (each \texttt{World} is associated with a
communicator) but contains all of the information necessary to provide the rich functionality of the MADNESS runtime.
Instead of calling \texttt{MPI::Init()} and \texttt{MPI::Finalize()}, a MADNESS program invokes
\texttt{madness::initialize()} and \texttt{madness::finalize()}
(call \texttt{madness::initialized()} to check that MADNESS had been
initialized). The routine \texttt{madness::startup()} initializes the MADNESS
numerical environment.

The only bit of voodoo is a macro, on the first line, forcing MADNESS to instantiate the static data associated with the
C++ templates used by your code. It typically only needs to appear in the main source file of your program, though
complicated projects might need it in additional files. 

\subsection{Step 4 -- generating a numerical representation}
We implemented the C++ function above and the domain is specified by the problem as  $x\in [0,10]$. Presently, all
functions of a given dimension being used by a program are assumed to have the same, default domain. We will use the
default precision (1e-6 in the L2-norm -- most errors in MADNESS use this norm).

\begin{quote}
	\begin{verbatim} 
	FunctionDefaults<1>::set_cubic_cell(0,10);
	
	real_function_1d f = real_factory_1d(world).f(myf);
	\end{verbatim}
\end{quote}	

The first line sets the domain and the second makes a numerical representation of your function. The slightly unusual
form of the constructor for a MADNESS function is called the ``named parameter idiom.'' The first version of MADNESS
was written in the language Python that provides named parameters that can have default values. To override the default
for some parameter only required specifying its name. However, C++ does not provide named parameters and the closest we
can get to them is the above idiom. To make a \texttt{Function} what you are actually doing is making a
\texttt{FunctionFactory} associated with a \texttt{World}, overriding defaults inside the factory, and then
constructing a function from the factory. The default function is zero, so to make a zero 1D function we would just
type
\begin{quote}
	\begin{verbatim} 
	real_function_1d zero = real_factory_1d(world);
	\end{verbatim}
\end{quote}	

In the second line, to generate a MADNESS \texttt{Function} from your C++ function we override the default with the
desired function pointer.

\subsection[Step 5 {}-- computing the integral]{Step 5 -- computing the integral}
\label{bkm:RefHeading39623517}The integral of a function over the whole domain is computed with the \texttt{trace()}
function, c.f.,
\begin{quote}
	\begin{verbatim} 
	double integral = f.trace();
	\end{verbatim}
\end{quote}	

\textit{Important:} This is a collective parallel operation. If you are running in parallel, the numerical
representation of your function is distributed across multiple processors, so adding up the result requires collective
communication (the equivalent of MPI's \texttt{All\_reduce()}). Therefore, every process associated with the
\texttt{World} must execute this statement; otherwise your program will hang if it is running in parallel using MPI.

\subsection[Step 6 {}-- printing the result]{Step 6 -- printing the result}
If you only want to run sequentially, just print your data in any of the myriad ways possible using C++. However, if you
ever want to run in parallel across multiple nodes using MPI you should from the outset get in the habit of writing the
following instead

\begin{quote}
	\begin{verbatim} 
	if (world.rank() == 0) print("The result is ", integral);
	\end{verbatim}
\end{quote}

The critical ingredient is the if-test that ensures only one process actually does the printing (in this case process
zero, but that choice is arbitrary) . If you don't do this, your output will be unexpectedly verbose. The templated
routine \texttt{print()} is just a convenience wrapper around \texttt{cout {\textless}{\textless}} that automatically
inserts spaces between elements and a newline at the end.

\textit{Important:}\textbf{ }First, read the important point from section \ref{bkm:RefHeading39623517} above, and then
answer the question ``why did we assign the result of \texttt{f.trace()} to a variable and then print its value,
instead of just printing the result of \texttt{f.trace()} directly?'' At least 90\% of hanging parallel programs is due
to getting this wrong.

\subsection{The complete program}
Your complete program, \texttt{trunk}\texttt{/src/apps/sininteg.cc, }looks like this

\begin{quote}
	\begin{verbatim} 
		#define WORLD_INSTANTIATE_STATIC_TEMPLATES  
		#include <madness/mra/mra.h>
		using namespace madness;
		
		double myf(const coord_1d& r) {
		    return std::sin(r[0]);
		}
		
		int main(int argc, char** argv) {
		    initialize(argc, argv);
		    World world(MPI::COMM_WORLD);
		    startup(world,argc,argv);
		    FunctionDefaults<1>::set_cubic_cell(0,10);
		    
		    real_function_1d f = real_factory_1d(world).f(myf);
		    
		    doub	le integral = f.trace();
		    
		    if (world.rank() == 0) print("The result is", integral);
		    finalize();
		    return 0;
		}
	\end{verbatim}
\end{quote}

Yes, this is rather verbose for integrating  $\sin (x)$ in 1D, but you should already be able to see that most of it is
boiler plate. You could be integrating a much more complicated function in 4-D with only little more personal effort
(the computer might have to work a lot harder though).

\subsection{Step 8 -- compiling your program}
Assuming your code is in saved in \texttt{fred.cc}, to make the program type

{\ttfamily
make -f trunk/config/Makefile.sample fred}

The example makefile \texttt{Makefile.sample} must be copied and modified for projects with more than one source file
(see the end of that file for an example).

\subsection{Step 9 -- running your program}
\subsubsection{On a single workstation or laptop}
Just type the name of the executable in the directory where the binary is located,

{\ttfamily
./fred}


\bigskip

Hopefully it will print

{\ttfamily
The result is 1.839072e+00}


\bigskip

\textit{You are already running in parallel. }Behind the scenes, MADNESS inquired how many processors or cores were
installed in your computer and launched that many threads to perform the work. If you wish to use fewer processors, set
the environment variable \texttt{MAD\_NUM\_THREADS} to be the total number of processors or cores you wish to use (note
that if using MPI an additional server thread is also employed for massage passing). E.g., to use just 3 processors, in
the bash shell type

{\ttfamily
export MAD\_NUM\_THREADS=3}

and in the C shell type

{\ttfamily
setenv MAD\_NUM\_THREADS 3}


\bigskip

\textit{Important: }Using more threads than processors can result in catastrophic slowdown since MADNESS uses spinlocks.
If you need to oversubscribe the physical processors (or run inside a virtual machine), configure with the option
\texttt{{}--enable-never-spin}.

\subsubsection{On a parallel computer using MPI (distributed or shared memory)}
This is going to vary from system to system. Generically, to run on \texttt{N} nodes with \texttt{T} applications
threads per node (an extra server thread for message passing is also created if \texttt{N}{\textgreater}1) , you will
type (in the bash shell) something like

{\ttfamily
MAD\_NUM\_THREADS=T mpirun -np N ./fred}


\bigskip

We want \texttt{mpirun} to start one process per node, and MADNESS will look after starting one thread per
processor/core. But this is easier said than done and sadly changes with each revision of the vendors system software.
Please refer to the MADNESS wiki for current examples for how to run on the Cray-XT5 and IBM BG/P.

\section{Defaults for functions}
Default values for all function and operator attributes are stored in the \ \texttt{FunctionDefaults} class. This is
actually a template so that different values can be set for functions with different numbers of dimensions. We saw
above that

{\ttfamily
FunctionDefaults{\textless}1{\textgreater}::set\_cubic\_cell(0,10);}

sets the user's simulation cell as  $[0,10]$. Presently, all functions of a given dimension must share the same cell.
Two other common attributes are

\begin{itemize}
\item \texttt{k} -- the wavelet order. A practical rule of thumb is if the default truncation threshold is  $10^{-n}$
that the order be chosen as  $k=n+2$. The default is 6.
\item \texttt{thresh} -- the truncation threshold. The default is 1e-4.
\item \texttt{bc} -- the boundary conditions. See below for more details. The default is free.
\end{itemize}
\section{Boundary conditions}
In MADNESS, boundary conditions are associated with operators not functions and the boundary conditions are imposed on
the surface enclosing the entire simulation volume. I.e., they are exterior boundary conditions. For derivative
operators the following conditions are understood, and can be imposed separately on each surface

\begin{itemize}
\item \texttt{BC\_ZERO} -- Zero Dirichlet
\item \texttt{BC\_PERIODIC} -- Periodic (both left and right surfaces must agree on this value)
\item \texttt{BC\_FREE }{}-- Free (default)
\item \texttt{BC\_DIRICHLET} -- General Dirichlet (requires provision of one or more functions)
\item \texttt{BC\_ZERONEUMANN }{}-- Zero Neumann
\item \texttt{BC\_NEUMANN} -- General Neumann (requires provision of one or more functions)
\end{itemize}
For integral operators only periodic and free-space conditions are understood -- BC\_PERIODIC yields periodic and all
other conditions yield free-space.

Example: to make the default boundary conditions in 3D

{\ttfamily
BoundaryConditions{\textless}3{\textgreater} bc(BC\_FREE);}

Example: to make boundary conditions in 3D with zero Dirichlet in x and y and periodic in z,

{\ttfamily
BoundaryConditions{\textless}3{\textgreater} bc(BC\_ZERO);}

{\ttfamily
bc(2,0) = bc(2,1) = BC\_PERIODIC}

Example: to override the default boundary conditions with yours in a variable named \texttt{bc} in 3D

{\ttfamily
FunctionDefaults{\textless}3{\textgreater}::set\_bc(bc);}

\section{Differentiation, multiplication, inner products -- e.g., energy of the hydrogen atom}
In this section we examine operations such as differentiation, inner product and multiplication. A relevant simple
example is \texttt{hatom\_energy.cc}.

\subsection{Differentiation}
Differentiation is performed by applying a differential operator to a function. The operator is constructed with desired
the boundary conditions and direction for differentiation (directions are indexed starting from zero, so in 3D
\texttt{x=0}, \texttt{y=1} and \texttt{z=2}). The operators can be kept for repeated application, or made and discarded
after use.

For example, to make the derivative operator in 3D with respect to the first variable using boundary conditions from
\texttt{FunctionDefaults}, and to apply it to functions \texttt{f}, \texttt{g} and \texttt{h}.

\begin{quote}
\begin{verbatim}
real_derivative_3d Dx(world, 0); 
real_function_3d dfdx = Dx(f); 
real_function_3d dgdx = Dx(g); 
real_function_3d dhdx = Dx(h); 
\end{verbatim}
\end{quote}

\subsection{Multiplication, addition, subtraction of functions}
Most simple mathematical operations can be composed in MADNESS as they are normally written in standard notation. For
instance, if \ \texttt{f}, \texttt{g} and \texttt{h} are functions the expression 

\begin{equation}
f(x)=2g(x)+3h(x)-7g(x)h(x)+99
\end{equation}
is transcribed as

{\ttfamily
f = 2*g + 3*h - 7*g*h + 99}

where \texttt{*} indicates point-wise multiplication of functions.

\textit{Important: }Addition and subtraction of functions are exact operations in the sense that the result can be
exactly represented in MADNESS basis. Multiplication is \textit{inexact} since the product of two polynomials of order 
$k$ is of order  $2k$. The auto-refinement algorithm within MADNESS is still under development -- please refer to the
implementation notes for more detail.

\subsection{Inner products}
The inner product of two functions is defined as 
\begin{equation}
\left( f \left| g \right. \right) - \int f(x)^\textrm{*} g(x) dx
\end{equation}
where  $\textrm{*}$ indicates complex conjugation and the integral is taken over the entire simulation volume. The above
is computed for two MADNESS functions \texttt{f} and \texttt{g} of the same type using

{\ttfamily
inner(f,g)}

If the input functions are real, the result is real; for complex functions the result is complex.

\section{Integral operators -- e.g., solution of Laplace or Poisson's equations}
The Poisson equation
\begin{equation}
\nabla ^{2}u=-4\pi \rho 
\end{equation}
is ubiquitous in scientific and engineering simulations. \ For the sake of simplicity, we assume free-space boundary
condition (zero at infinity, so the Green's function is just  $1/\left|r\right|$). \ \ If the right hand side of the
Poisson equation is \texttt{rho} then the Poisson equation can be solved as

{\ttfamily
real\_convolution\_3d op=CoulombOperator(world, 0.001, 1e-6);}

{\ttfamily
real\_function\_3d result = op(rho);}

This is employed in many example codes in the examples directory. The call to the Coulomb operator builds a
low-separation rank approximation (see the implementation notes) of the Green function for the Poisson equation. \ The
approximation is accurate to 1e-6 from a smallest length scale of 0.001 to the entire box size.

If you have more complicated boundary conditions which require single or double layer terms please refer the examples in
\texttt{interior\_dirichlet.cc} or \texttt{laplace\_sphere.cc} for more details.

\section{Operations on vectors of functions -- e.g., calculating multiple eigenfunctions}
The header file \texttt{vmra.h} defines operations on vectors of functions. These are convenient in eliminating
error-prone loops over arrays/vectors of functions, and the vector operations are much more efficient since many
operations can occur in parallel. The example code \texttt{vnucso.cc} and the molecular density function code make
extensive use of the vector API (application programming interface) to solve eigen-problems.

Given a subspace defined by a vector of  $n$ functions  $f_{i}(x)\ i=0,\ldots ,n-1$ we can diagonalize the operator 
$\hat{H}$ in the subspace by constructing the matrix representations of the operator ( $H$) and metric ( $S$)
\begin{align*}
h_{ij} & = \left< f_i \left| \hat{H} \right| f_j \right> \\
s_{ij} & = \left< f_i \left|         \right. f_j \right>
\end{align*}
and then solving the generalized eigenvalue problem 

\begin{equation}
HC=SCE
\end{equation}
to obtain the eigenvalues and coefficients in the subspace. The eigenfunctions  $u_{i}(x)$ are obtained by transforming
the original basis 

\begin{equation}
u=fC\ \ \ \mathrm{\text{or}}\ \ \ u_{i}(x)=\sum _{j}f_{j}(x)c_{ji}
\end{equation}

\bigskip

Given an STL vector of 3D functions \texttt{f} and another \texttt{Hf} containing result of applying the operator 
$\hat{H}$ to the vector, the above is compactly translated into MADNESS as

\begin{quote}
\begin{verbatim}
real_tensor H = matrix_inner(f,Hf); 
real_tensor S = matrix_inner(f,f); 
real_tensor C,E; 
sygv(H,S,1,C,E); 
vector_real_function_3d evec = transform(f,C); 
\end{verbatim}
\end{quote}

The \texttt{matrix\_inner()} routine computes the matrix of inner products (or matrix elements) of two vectors of
functions, and the \texttt{sygv()} routine (in \texttt{linalg/tensor\_lapack.h}) is a wrapper around the LAPACK real
symmetric and complex Hermitian generalized eigenvalue routines. Finally, the \texttt{transform()} routine transforms
the basis to compute the eigenfunctions.

\section{Moving functions to/from disk}
It takes at most a couple of lines of code to move a function to/from disk. \ The example program \texttt{functionio.cc}
shows how to do this, and the critical lines are
\begin{quote}
\begin{verbatim}
ParallelOutputArchive out(world, filename); 
out & f & g & h; 
ParallelInputArchive in(world, filename); 
in & f & g & h; 
\end{verbatim}
\end{quote}

Here, \texttt{f}, \texttt{g}, and \texttt{h} are MADNESS functions of any type (they need not have the same type or
dimension). 

MADNESS has a generic mechanism for (de)serializing objects from directional streams that it calls archives to avoid
confusion with the STL stream concept. \ \ \ Anything can be written to and read from a stream, though user-defined
types usually need a little additional code to enable this (see \texttt{world/archive.h} for documentation). Most
archives are sequential objects accessible only to a single process and it makes little sense to write a potentially
large function that is distributed across the entire parallel computer to such a stream. For that reason, functions
must be written to/from a parallel archive that performs efficient parallel I/O.

Large amounts of data (more than a few gigabytes) will benefit from increasing the number of processors actually doing
disk I/O (refer to the relevant parallel archive class for more info).

Note that the program \texttt{mraplot} can be used to generate plots from functions stored on disk.

\section{Plotting}
MADNESS can presently generate uniform grids in formats suitable for the visualization software OpenDX (in a
``\texttt{.dx}{}'' file) and Paraview (in a VTK file), and line plots in a text format suitable for nearly any tool
(e.g., \ gnuplot or xmgrace).

Note that the program \texttt{mraplot} can be used to generate plots from functions stored on disk.

\subsection{OpenDX}
OpenDX is an Open Source visualization software based on IBM's Visualization Data Explorer
(\href{http://www.opendx.org/}{www.opendx.org}). \ Please refere to the OpenDX manual on its use. \ \ An example
configuration file (\texttt{vizit.cfg}) \ and a network file (\texttt{vizit.net}) can be found in the
\texttt{trunk/src/apps/moldft} directory. \ Here are some details on creating your own DX files.

Given any MADNESS function \texttt{f}, you can write to disk a uniform grid over the entire simulation volume as follows

{\ttfamily
plotdx(f, ``f.dx'');}


\bigskip

Additional arguments permit us to change the plot volume (default is the entire simulation volume), the number of points
(default is 201) and binary/text format (default is binary). This is a collective operation.

Visualizing this with OpenDX is straightforward, but depends on the number of dimensions. The easiest way is to start
OpenDX, click ``Import Data ...'', enter your filename, click ``Visualize Data''. However, you will want to learn how
to build your own networks. To display an iso-surface of 3D data, start the visual editor and connect

{\ttfamily
file-selector -{}-{\textgreater} import -{}-{\textgreater} isosurface -{}-{\textgreater} image}

Enter your name in the file-selector, and you should be seeing a picture. You can adjust the isosurface value in the
isosurface control or by connecting it to an interactor. Have fun!

\subsection{Line plots}
With a single function call, up to three functions can be simultaneously evaluated at points along a line with the
values printed to a file suitable for use by any standard 2D graphics or spreadsheet software. E.g., to plot one 4D
function (\texttt{f}) along a line between (0,0,0,1) and (0,1,0,2) with 101 points

\begin{quote}
\begin{verbatim}
coord_3d lo,hi; 
lo[3]=1.0; hi[1]=1.0; hi[2]=2.0; 
plot_line(��plot.txt��, 101, lo, hi, f); 
\end{verbatim}
\end{quote}

To plot two functions (\texttt{f} and \texttt{g}) you would use instead

{\ttfamily
plot\_line(``plot.txt'', 101, lo, hi, f, g)}

With gnuplot, you can plot the data as follows

\begin{quote}
\begin{verbatim}
gnuplot -persist -e 'set style data lines; plot "plot.txt"'
\end{verbatim}
\end{quote}

\subsection{VTK format and Paraview}
MADNESS can export MADNESS functions to the serial vtkStructuredGrid (.vts) file format which can be ready by several
post-processing visualization packages such as Paraview. \ To write this data file, you must first define four
quantities: a filename (\textit{filename)}, the lower (\textit{plotlo}) and upper (\textit{plothi}) bound of the
Cartesian coordinates in each dimension to plot, and the number of points (\textit{npts)} in each dimension to
evaluate.

\textit{Note:}\textbf{ }At this time, MADNESS evaluates the functions at equally spaced points between \textit{plotlo
}and \textit{plothi}. \ 

After the above four quantities have been defined, three functions must be called:

\ \ \texttt{plotvtk\_begin }: Writes the VTK boilerplate header information

\ \ \texttt{plotvtk\_data} : Evaluates and writes the MADNESS function at \textit{npts}

\ \ \texttt{plotvtk\_end }: Writes the VTK boilerplate footer information


\bigskip

\textit{Note:}\textbf{ }Currently, the function evaluation in plotvtk\_data is done in serial. \ 

An example of code for plotting two two-dimensional functions \textit{u} and \textit{v} is as follows in a
[0,1]\textsuperscript{2} box with 101 points is as follows:

\begin{quote}
\begin{verbatim}
char filename[100]; 
sprintf(filename, output.vts); // Defines the filename 
Vector<double, 2> plotlo, plothi; // Box dimensions 
Vector<long, 2> npts; // Num points in each dim for(int 
i = 0; i < 2; ++i) 
{ 
	plotlo[i] = 0.0; 
	plothi[i] = 1.0; 
	npts[i] = 101; 
} 
plotvtk_begin(world, filename, plotlo, plothi, npts); 
plotvtk_data(u, u, world, filename, plotlo, plothi, npts); 
plotvtk_data(v, v, world, filename, plotlo, plothi, npts); 
plotvtk_end<3>(world, filename); 
\end{verbatim}
\end{quote}

\textit{Note: }An arbitrary number of MADNESS files may be written to a single file. \ Two are shown above for
demonstrative purposes; however, plotvtk\_data may be called multiple times between plotvtk\_begin and plotvtk\_end
function calls.

{\bfseries
\textmd{\textit{Note: }}\textmd{For time-dependent simulations, the above code can be included within the time step
loop. \ The user should then consider appending the filename with the timestep number and create Ntimestep files, each
containing one increment of time.}}

To visualize your functions, Paraview has been extensively tested, although other external visualization packages may
also be compatible with the vts file format. \ Paraview is an open-source visualization application that is freely
downloadable. \ For information about how to download, install and use Paraview, please consult their webpage at
http://www.paraview.org.

\section{Load and memory balancing}
Load and memory balancing is a critical issue on current generation of shared and distributed memory computers. \ On
many terascale and petascale computer there is no virtual memory capabilities on the compute nodes so memory management
is very important.

Poor distribution of work (load imbalance) is the largest reason for inefficient parallel execution within MADNESS.
\ Poor data distribution (data imbalance) contributes to load imbalance and also leads to out-of-memory problems due to
one or more processes having too much data. \ Thus, we are interested in uniform distribution of both work and data. 

Many operations in MADNESS are entirely data driven (i.e., computation occurs in the processor that
"owns" the data) since there is insufficient work to justify moving data between processes
\ (e.g., computing the inner product between functions). \ However, a few expensive operations can have work shipped to
other processors. 

There are presently three load balancing mechanisms within MADNESS 

\begin{itemize}
\item static and driven by the distribution of data, 
\item dynamic via random assignment of work, and 
\item dynamic via work stealing (currently only in prototype). 
\end{itemize}
Until the work stealing becomes production quality we must exploit the first two forms. \ The random work assignment is
controlled by options in the FunctionDefaults class. 

\begin{itemize}
\item \texttt{FunctionDefaults::set\_apply\_randomize(bool) }controls the use of randomization in applying integral
(convolution) operators. It is typically beneficial when computing to medium/high precision. 
\item \texttt{FunctionDefaults::set\_project\_randomize(bool)} controls the use of randomization in projecting from an
analytic form (i.e., C++) into the discontinuous spectral element basis. \ It is typically \ beneficial unless there is
already a good static data distribution. \ Since these options are straightforward to enable, this example focuses on
static data redistribution. 
\end{itemize}
The process map (an instance of \texttt{WorldDCPmapInterface}) controls mapping of data to processors and it is actually
quite easy to write your own (e.g., see \texttt{WorldDCDefaultPmap }\texttt{or LevelPmap}) that ensure uniform data
distribution. \ However, you also seek to incorporate estimates of the computational cost into the distribution. \ The
class \texttt{LBDeuxPmap} (deux since it is the second such class) in \texttt{trunk/src/lib/mra/lbdeux.h} does this by
examining the functions you request and using provided weights to estimate the computational cost. 

Communication costs are proportional to the number of broken links in the tree. \ Since some operations work in the
scaling function basis, some in the multiwavelet basis, and some in non-standard form, there is an element of
empiricism in getting best performance from most algorithms. 

The example code in \texttt{trunk/src/apps/examples/dataloadbal.cc} illustrates how the discussions in this section can
be applied.

\section{How to ``think'' MADNESS}
MADNESS is based on multiresolution analysis (MRA) and low-separation rank (LSR) approximations of functions and
operators. \ It can be considered an adaptive spectral elemen method using a discontinuous and singular multiresolution
basis. \ The representations of differential operators in the wavelet bases are provably similar to adaptive finite
difference discretizations. \ Thus, the process of solving the resulting linear systems will have similar behaviors as
in other adaptive methods. \ For example, the derivative operator and the Laplacian operator are unbounded operators.
\ Thus the condition number, which often constraints how accurately the linear system can be solved, goes to infinity
as the bases or the nets are refined. \ In order to solve these equations in practice, one has to precondition the
system. \ Effective preconditioners are problem dependent and the theory of their construction is an area of on-going
research.

The integral operator, which is the formal inverse associated to the differential operator, are usually bounded
operator. \ MRA and LSR have been proven to be one of the techniques that can effectively apply some of the physically
important operators and their kernel fast and with ease.

Two of the most important operators that we illustrate in this manual are the Poisson operator (in section 5), and the
\ Helmholtz operator. \ The heat equation is ...

\subsection{Solve the integral equation}
In many situations the integral operator associated to the differential operator have an analytic kernel. \ The simplest
examples are the convolution operators. \ In equation (9) the free-space Poisson equation is converted to a convolution
with the Poisson (or Coulomb) kernel, the Schrodinger equation with potential  $V$ is converted to a Lippman-Schwinger
equation using the bound-state Helmholtz kernel, and the last example applies Duhamel's principle to write a time
dependent differential equation with linear operator  $\hat{L}$ and a non-linear operator  $N$ as a semi-group
equation.


\begin{align*}\label{eqn:refText8}
\nabla ^{2}u=-4\pi \rho                     & \to u=G\ast \rho    \ \ \    \textrm{where} G(r-r')=\frac{1}{|r-r'|} \\
-{\frac{1}{2}}\nabla ^{2}\psi +V\psi =E\psi & \to \psi =-2G\ast V\psi \ \ \ \textrm{where} G(r-r')=\frac{e^{-\sqrt{-2E|r-r'|}}}{4\pi |r-r'|}\\
\hat{L}u+N(u,t)=\dot{u}                     & \to u(t)=e^{\hat{L}t}u(0) +\int _{0}^{t}e^{\hat{L}(t-t')}N(u,t')\mathit{dt}' &
\end{align*}

Most codes, including MADNESS, are bad at solving differential equations to high accuracy -- this is why there is so
much emphasis placed on finding a good pre-conditioner. The problem arises from the spectrum of the differential
operator. Consider the Laplacian in 1D acting on a plane wave 

\begin{equation}
\frac{d^{2}}{\mathit{dx}^{2}}e^{i\omega x}=-\omega ^{2}e^{i\omega x}
\end{equation}
The effect of the Laplacian is to greatly amplify high frequencies  $\omega $ (where most of the numerical error lies)
whereas physical applications are primarily interested in lower frequencies. The eigenvalues of the corresponding
inverse or integral operator have the opposite effect -- high frequencies are suppressed and lower frequencies
emphasized.

The integral form is potentially better in many ways -- accuracy, speed, robustness, asymptotic behavior, etc.. If you
really, really, want to solve the differential form, then instead of using the phrase ``integral form'' say ``perfectly
preconditioned differential form'' so that you can do the right thing.

\subsection{Carefully analyze discontinuities, noise, singularities, and asymptotic forms}
Your function needs to be evaluated at close to machine precision. \ The higher the order of the basis ( $k$) the
greater the necessary accuracy regardless of what threshold you are trying to compute to. \ The accuracy and
convergence of the Gauss-Legendre quadrature rests on the function being smooth (well approximated by a polynomial) at
some level of refinement. Discontinuities in the function value or derivatives, \ singularities, and numerical noise
can all cause excessive refinement as MADNESS tries to deliver the requested precision-Gibbs effect in action.
\textit{The usual symptoms of this problem are unexpectedly slow execution and excessive memory use.} Here are some
tips to work with these effects.

Discontinuities and singularities need to be consciously managed. Integrable point singularities might sometimes work
unmodified (e.g.,  $1/r$ in 3-D) but can unpredictably fail, e.g., if a quadrature point lands very near the
singularity by accident. If possible, arrange for such points/surfaces to coincide with dyadic points (i.e., an integer
multiple of some power of two division of the domain) -- this will give the most accurate representation and exploits
the discontinuous spectral basis. If you cannot ensure such placement, you must manually or analytically regularize the
function and one would usually employ a parameter to control the length scale of any problem modification and to enable
systematic demonstration of convergence. E.g., eliminate the cusp in an exponential with

\begin{equation}
\exp (-r)\to \exp (-\sqrt{r^{2}+\sigma ^{2}})
\end{equation}
or replace a step function with 

\begin{equation}
\theta (x)\to \theta (x,\lambda )=\frac{1}{2}(1+\tanh \frac{x}{\lambda })
\end{equation}
or the Coulomb potential in 3-D with 

\begin{equation}
\frac{1}{r}\to u(r,c)=\frac{1}{r}\operatorname{erf}\frac{r}{c}+\frac{1}{c\sqrt{\pi
}}e^{-\left(\frac{r}{c}\right)^{2}}\ \ \ \ \ \ \ \ \ \ \ \ \ \ \int _{0}^{\infty
}\left(u(r,c)-r^{-1}\right)r^{2}\mathit{dr}=0
\end{equation}
(the integral indicates that the mean error is zero independent of  $c$).

Numerical noise can be a problem if your function is evaluated using interpolation or some other approximation scheme,
or when switching between representations (e.g., between forms suitable for small or large arguments). If you are
observing inefficient projection into the basis, ensure that your approximation is everywhere smooth to circa 1 part in
 $10^{12}$ or better.

MADNESS itself computes to a finite precision and when computing a point-wise function of a function (i.e.,  $g(f(x))$
where  $f(x)$ is a MADNESS function and  $g(s)$ is a user-provided function) the user-provided function must tolerate
that approximation within tolerance or noise. A classic example is computing the function

\begin{equation}
V(\rho (x))=\frac{C}{\rho ^{1/3}(x)}
\end{equation}
where in the original problem one knows that  $\rho (x)>0,\ \forall x$ but numerically this positivity not guaranteed.
In this case an effective smoothing is 

\begin{equation}
\begin{gathered}V(\rho )\to V(S(\rho ))\\S(s)=\left\{\begin{matrix}s_{0}&s\le 0\\q(s,s_{0,}s_{1})&0<s\le
s_{1}\\s&s>s_{1}\end{matrix}\right.\\q(s,s_{0},s_{1})=s_{0}-(-2s_{1}+3s_{0})\left(\frac{s}{s_{1}}\right)^{2}+(2s_{0}-s_{1})\left(\frac{s}{s_{1}}\right)^{3}\end{gathered}
\end{equation}
The function  $S(s)$ coincides with its argument for  $s>s_{1}$ and for smaller values smoothly switches to a minimum
value of  $s_{0}$ with continuous value and derivative at both end points.

Some computations are intrinsically expensive. For instance, the function  $\exp (i\omega r)$ is oscillatory everywhere
and the number of required coefficients will increase linearly with the solution volume. \ In a 3-D box of width  $L$,
the number of coefficients will be  $O\left(\left(Lk\omega \right)^{3}\right)$ (where  $k$ is the multiwaveler or
polynomial order). For  $L=1000$,  $k=12$ and  $\omega =3$, a few hundred TB of data (i.e., too big!) will be
generated. Thus, it is worth making a back of the envelope estimate about the expected cost of computation before
getting started.

\subsection[Robustly trade precision for speed]{Robustly trade precision for speed}
TO BE COMPLETED.

\subsection{Choice of polynomial order (k)}
TO BE COMPLETED.

\section{Environment variables}
\texttt{MAD\_BIND} -- Specifies the binding of threads to physical processors. On both the Cray-XT and the IBM BG/P the
default value should be used. On other machines there is sometimes a small performance gain to be had from forcing
threads to use the same processor, thereby improving cache locality. The value is a character string containing three
integers in the range. The first indicates the core to which the main thread should be bound, the second the core for
the communication thread, and the third the core for first thread in the pool. Subsequent threads use successively
higher cores. A value of -1 indicates do not bind. The default on the XT is ``\texttt{1 0 2}{}'' and on the BG/P
``\texttt{{}-1 -1 -1}{}''.

\texttt{MAD\_NUM\_THREADS} -- Specifies the total number of threads to be used by each MPI process. If running with just
one MPI processes, there will be this many threads executing the application code so the minimum value is one. If
running with more than one MPI processes, one thread is dedicated to communication so the minimum value is two. The
default value is the number of processors detected (using this default is the only way presently to have different
numbers of threads on different nodes).

\texttt{MRA\_DATA\_DIR} -- Specifies the directory that contains the MADNESS data files (notably the autocorrelation
coefficients, two-scale coefficients, and Gauss-Legendre points and weights). Sometimes the compiled-in default must be
overridden. Only MPI process zero will use this.

\texttt{POOL\_NTHREAD} (deprecated) -- Specifies the number of threads in the compute pool. New users should set
\texttt{MAD\_NUM\_THREADS} instead.

\section{Troubleshooting.}
TO BE COMPLETED.

\section{Viewing and editing this document}
All of our documents use the same styles and fonts -- consistency is important! Do not change the styles or fonts.

\begin{itemize}
\item Use OpenOffice 3.1 or later to edit this file.
\item Under Linux ensure you have the Microsoft true-type fonts installed -- they are free. Under Ubuntu install package
\texttt{msttcorefonts}. Without these the default Linux fonts will cause pagination and other problems, at least with
the title pages.
\item All text except for foreign languages and symbols should be in Times New Roman.
\item All in-line code, filenames, commands, etc., should be in Courier New.
\item All multi-line code examples should be in Code style (note that it automatically indents).
\item Emphasis should be done with italics -- bold should not be used.
\item Use the default bullets, numbering, etc.
\end{itemize}

\bigskip
\end{document}
