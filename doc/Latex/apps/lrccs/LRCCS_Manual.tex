\documentclass[10pt,a4paper]{scrartcl}

\usepackage{amsmath}
\usepackage{amsfonts}
\usepackage{amssymb}
\usepackage{color}
\usepackage[english]{babel}
\usepackage[
	colorlinks=true,
	urlcolor=blue,
	linkcolor=green
]{hyperref}

% no indentation
\setlength{\parindent}{0pt} 

\begin{document}
\section*{MADNESS: CIS Excitation Energies}
The Tutorial was written for the Git commit \textit{f03a51fbe4f5a30d870c28772e5960ae39502da9} which you can checkout before compiling via: \\
\textit{git checkout f03a51fbe4f5a30d870c28772e5960ae39502da9}\\
If you experience any sort of trouble please feel free to contact me at:\\
\href{mailto:jakob.kottmann@hu-berlin.de}{jakob.kottmann@hu-berlin.de}

\section{Built LRCCS}
Detailed information can be found in the MADNESS wiki on Github: \url{https://github.com/m-a-d-n-e-s-s/madness/wiki}
\begin{itemize}
\item Make shure you have Git installed
\item Make shure you have (see also the MADNESS wiki):
\begin{itemize}
\item C++11 Compiler: Clang-3.8 or later (Mac OS), GCC-4.8 or later (Linux)
\item Corresponding MPICH
\item GNU Autotools (autoconf 2.59 or later, automake 1.11 or later)
\end{itemize}
\item Get the MADNESS source Code from Github\\
\textit{git clone {https://github.com/m-a-d-n-e-s-s/madness.git} madness-source}
\item Execute \textit{autogen.sh} in the madness-source directory
\item Configure MADNESSS in your compile directory like this (use the correct paths):\\
\textit{/path-to-source/madness-source/configure 'CC=/usr/bin/clang' 'CXX=/usr/bin/clang++'
'MPICC=/opt/local/bin/mpicc-mpich-clang38' 'MPICXX=/opt/local/bin/mpicxx-mpich-clang38'} 
\item Compile the MADNESS libraries in your compile directory\\
\textit{make -C /path-to-madness-compile/madness-compile/ libraries}
\item Compile LRCCS in the MADNESS examples directory\\
\textit{make -C /path-to-madness-compile/madness-compile/src/examples lrccs}
\end{itemize}

\section{Run LRCCS}
After compiling you can execute \textit{lrccs} in the \textit{madness-compile/src/examples} directory. For the calculation it is necessary that an input file with the name \textit{input} is present in the directory from which you call lrccs.

\clearpage
\section{Quick Tutorial}
The programm \textit{LRCCS} \cite{mra-cis} will solve the linear-response equations for excitation energies in the CIS approximation via the following steps.
\begin{enumerate}
\item Solve Hartree-Fock (HF) ground state
\item Create guess vectorfunctions $\lbrace x_i \rbrace $ from the HF orbitals $\lbrace \phi_i \rbrace$ via
\[ x_i = f\cdot\phi_i \]
where $f$ is constructed from polynomials (see Keyword section)
\item Iterate the guess vectorfunctions 
\item Iterate the lowest guess vectorfunctions further
\item Iterate sequentially untill convergence
\end{enumerate}
Detailed information can be found in \cite{mra-cis}.
\subsection{Input File Structure}
The input file has four main sections.
\paragraph{Ground-State section:} Begins with \textit{dft} and ends with \textit{end}\\

Mandatory Keywords:
\begin{itemize}
\item xc value: \textit{value} is the exchange correlation potential. To solve the CIS equations this has to be always \textit{hf} 
\item k value: \textit{value} is the order of polynomials for the MRA representation (usually around 6-9)
\item econv value: \textit{value} is the MRA:threshold for the HF orbitals and at the same time the convergence threshold for the energy
\end{itemize}
Optional Keywords:
\begin{itemize}
\item L value: \textit{value} it the size of the cubic simulation box in atomic units. Each dimension of the box runs rum -L to L (default is 50.0)
\item dconv value: \textit{value} convergence threshold for the orbitals (does not affect MRA threshold)
\item nuclear\_corrfac value: \textit{value} is the nuclear correlation factor which is used \\
(default is \textit{none}, other options are \textit{slater}, \textit{GradientalGaussSlater}, \textit{GaussSlater}, \textit{LinearSlater} and
	    \textit{Polynomial}). For more information see \cite{nemo-I}.
\item no\_compute or restart: Load the HF equations from the file \textit{restartdata.00000} which has to be present in the same directory. If \textit{restart} is chosen the HF orbitals will be re-iterated
\end{itemize}

\paragraph{Response Section:} Begins with \textit{cc2} and ends with \textit{end}\\

Mandatory Keywords
\begin{itemize}
\item thresh\_3D value: \textit{value} is the MRA threshold for the response vectorfunctions (default is 5)
\end{itemize}
Optional Keywords
\begin{itemize}
\item tda\_econv\_hard value: \textit{value} is the final convergence threshold for the excitation energies
\item tda\_dconv\_hard value: \textit{value} is the final convergence threshold for the response vectorfunctions
\item tda\_excitations value: \textit{value} is the number of excitation energies which shall be calculated
\item tda\_guess\_excitations value: \textit{value} is the number of guess excitations which is iterated in the beginning
\item tda\_iterating\_excitations value: \textit{value} is the number of excitation vectors which are iterated together (change if you have memory problems)
\item freeze value: \textit{value} is the number of frozen core orbitals
\item tda\_guess value: \textit{value} is the guess that is used (default is a big guess from a perturbed fock matrix). Possible entries are: \textit{dipole}, \textit{dipole+}, \textit{quadrupole}, \textit{big\_fock\_3}, \textit{big\_fock\_4}, \textit{c2v}, \textit{c2v\_big} and \textit{custom}).
The polynomials which are used for the specific guesses will be displayed in the output. The \textit{custom} valueword has to be used together with the \textit{exop} valueword.
\item exop value: \textit{value} is a custom polynomial to create a guess vectorfunction from the HF orbitals.
Example: \textit{exop x 1.0 y 2.0, x 3.0 z 1.0 c -2.0} corresponds to $xy^2 + -2.0\cdot x^3z$.
To calculate more than one guess the exop keyword can used more than one time.
Example: The following line would be equivalent with the \textit{tda\_guess dipole+} keyword: \\
\textit{tda\_guess custom} \\
\textit{exop x 1.0}\\
\textit{exop y 1.0}\\
\textit{exop z 1.0}\\
\textit{exop x 2.0, y 2.0, z 2.0}
\end{itemize}
\clearpage
\paragraph{Geometry section:} Begins with \textit{geometry} and ends with \textit{end}. Contains the molecular coordinates in atomic units
\paragraph{Plot section:} Begins with \textit{plot} and ends with \textit{end}. Contains plotting information. Can in principle be empty but should be present. 
\subsection{Example Input}
The example input File should also be found in the madness-source/src/examples directory with the name \textit{input\_example\_lrccs}. Note that it has to be renamed to \textit{input}\\


dft\\
  xc hf\\
  econv 1.e-5\\
  dconv 1.e-4\\
end\\

cc2\\
  thresh\_3D 1.e-4\\
  tda\_econv\_hard 1.e-4\\
  tda\_dconv\_hard 1.e-3\\
  tda\_guess dipole+ \\
end\\

geometry\\
	he 0.0 0.0 0.0\\
end\\

plot \\
 plane x1 x2\\
 origin 0.0 0.0 0.0\\
 zoom 1.0\\
end

%References
\bibliographystyle{plain}
\bibliography{references}

\end{document}